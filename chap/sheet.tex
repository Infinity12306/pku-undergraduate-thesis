
\thispagestyle{empty}
\newgeometry{left=2cm, right=2cm, top=2.64cm, bottom=2.54cm}
\renewcommand\arraystretch{1.2}

\begin{center}
	{\songti\zihao{3}{北京大学本科毕业论文导师评阅表}}
\end{center}

%%%%%%%%%%%%%%%%%%%%%%%%%%%%%%%%%%%%%%%%%%%%%%%%%%%%%
%                                                   %
%                  请修改以下部分!                    %
%                                                   %
%%%%%%%%%%%%%%%%%%%%%%%%%%%%%%%%%%%%%%%%%%%%%%%%%%%%%

% 导师评语

\def\mentorcomment{\parbox[l]{35.88em}{
  \qquad
  哭过笑过恋过恨过,仿佛是一梦蹉跎。

  \qquad
  迷惑失落忧郁寂寞,谁都是凡人一个。

  \qquad
  细水还来不及长流,抽刀已经断不了情愁。

  \qquad
  牵手还是放手,不如一歌。

  \qquad
  在遗忘中不舍,醉醒交错,

  \qquad
  青春大概如你所说。
  
  \qquad
  在花落时结果,期望很多,青春大概都这样过。

  \qquad

  \qquad 评语部分会自动换行。两个连续的空行代表一个新的段落,以上《青春大概》歌词的分段就是这样完成的。
  导师签名部分相对于评语正文的结束位置的竖直距离是固定的,因此请控制导师评语的长度,使其能容纳在本页内。
  用户只需修改本部分的文字,而无需调整任何其他的参数。

  \qquad 本表格里的除导师评语部分外的可变信息都在主文件{\tt thesis.tex}以文档信息的方式定义。它们在封面和本表格中共享。
  例如,只要你在{\tt thesis.tex}中修改了论文的中文标题,那么此处表格里的内容将在下次编译时随封面标题同步更新。

  \qquad {\tt thesis.tex}内你应当只修改导师评语。
}}


%%%%%%%%%%%%%%%%%%%%%%%%%%%%%%%%%%%%%%%%%%%%%%%%%%%%%
%                                                   %
%                   不要改动以下部分!                 %
%                                                   %
%%%%%%%%%%%%%%%%%%%%%%%%%%%%%%%%%%%%%%%%%%%%%%%%%%%%%

\def\signatureplaceholder{\parbox[l]{16em}{
  \phantom{青春大概}
  
  导师签字:\phantom{青春大概}

  \phantom{青春大概}
}}

\begin{table}[H]
\centering
\begin{tabular}{|m{4em}m{4em}m{5em}m{6em}m{6em}l|}
\hline
\multicolumn{1}{|l|}{学生姓名}                  & \multicolumn{1}{m{4em}|}{\authorsname}                  & \multicolumn{1}{l|}{学生学号}                  & \multicolumn{1}{m{6em}|}{\mystudentid}                  & \multicolumn{1}{l|}{论文成绩}                  &       \myscore                  \\ \hline
\multicolumn{1}{|l|}{学院(系)}                  & \multicolumn{3}{l|}{\myschool}                                                                                                    & \multicolumn{1}{l|}{学生所在专业}                  &    \mycmajor                                    \\ \hline
\multicolumn{1}{|l|}{\multirow{2}{*}{导师姓名}} & \multicolumn{1}{l|}{\multirow{2}{4em}{\mycmentornotitle}} & \multicolumn{1}{l|}{\multirow{2}{5em}{导师单位/ 所在研究所}} & \multicolumn{1}{l|}{\multirow[]{2}{6em}{\mymentorinstitute}} & \multicolumn{1}{l|}{\multirow{2}{*}{导师职称}} & \multirow{2}{*}{\mycmentortitle}                      \\
\multicolumn{1}{|l|}{}                  & \multicolumn{1}{l|}{}                  & \multicolumn{1}{l|}{}                  & \multicolumn{1}{l|}{}                  & \multicolumn{1}{l|}{}                  &                                        \\ \hline
\multicolumn{2}{|c|}{\multirow{2}{*}{\shortstack{论文题目\\[0.35em](中、英文)}}} & \multicolumn{4}{c|}{\multirow{2}{*}{\shortstack{\myctitle\\[0.35em]\myetitle}}}  \\
\multicolumn{2}{|c|}{}                  & \multicolumn{4}{c|}{} \\ \hline
\multicolumn{6}{|m{36.88em}|}{\center{导师评语}} \\
\multicolumn{6}{|m{36.88em}|}{\centering\kaishu{(包含对论文的性质、难度、分量、综合训练等是否符合培养目标的目的等评价)}} \\
\multicolumn{6}{|m{36.88em}|}{} \\
\multicolumn{6}{|m{36.88em}|}{{\mentorcomment}} \\
\multicolumn{6}{|m{36.88em}|}{} \\
\multicolumn{6}{|m{36.88em}|}{\hfill\signatureplaceholder} \\
\multicolumn{6}{|m{36.88em}|}{\hfill 年 \qquad\quad 月 \qquad\quad 日 \qquad\qquad\qquad} \\
\multicolumn{6}{|c|}{} \\ \hline
\end{tabular}
\end{table}

\renewcommand\arraystretch{1}
\restoregeometry

